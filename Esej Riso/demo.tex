\documentclass{sig-alternate}
\usepackage[slovak]{babel}
\usepackage[utf8]{inputenc}
\usepackage[T1]{fontenc}
\usepackage{url}
\usepackage{booktabs}
\setlength{\heavyrulewidth}{1.2pt}
\setlength{\lightrulewidth}{0.7pt}

\begin{document}
\title{ESEJ}

\numberofauthors{2}

\author{
\alignauthor
TODO TODO\\
       \affaddr{TODO,}\\
       \affaddr{TODO,}\\
       \affaddr{TODO,}\\
       \affaddr{TODO}\\
       \email{TODO}
\alignauthor
TODO TODO\\
       \affaddr{TODO,}\\
       \affaddr{TODO,}\\
       \affaddr{TODO,}\\
       \affaddr{TODO}\\
       \email{TODO}
}

\maketitle

\section{Úvod}

Doplním úvod keď budeme hotoví so všetkým ostatným.

\section{Časť o pozornosti}

Možno ste si všimli, že internetové noviny sa veľmi líšia od papierových. Už len forma internetových novín je iná. Dôvodom je nový spôsob žurnalistiky.
Intenetových žurnalistov totiž platia inak ako klasických. Dostávajú príplatky podľa toho, koľko ľudí im na ich článok klikne. Takže cieľ novodobých internetových žurnalistov je zaujať čitateľa dostatočne aby na ich článok klikol. Teraz, keď máme internet zaplavený rôznymi novinami a článkami, ako môžeme spôsobiť aby práve ten náš vynikol? Jediná časť, ktorá je dostupná čitateľovi pred tým ako klikne na článok je jeho titul a žurnalisti sa v priebehu času naučili ho použiť.

%\bibliographystyle{abbrv}
%\bibliography{TODO}

\balancecolumns

\end{document}
